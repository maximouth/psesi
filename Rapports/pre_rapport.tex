%%%%%%%%%%%%%%%%%%%%%%%%%%%%%%%%%%%%%%%%%
% University Assignment Title Page 
% LaTeX Template
% Version 1.0 (27/12/12)
%
% This template has been downloaded from:
% http://www.LaTeXTemplates.com
%
% Original author:
% WikiBooks (http://en.wikibooks.org/wiki/LaTeX/Title_Creation)
%
% License:
% CC BY-NC-SA 3.0 (http://creativecommons.org/licenses/by-nc-sa/3.0/)
% 
% Instructions for using this template:
% This title page is capable of being compiled as is. This is not useful for 
% including it in another document. To do this, you have two options: 
%
% 1) Copy/paste everything between \begin{document} and \end{document} 
% starting at \begin{titlepage} and paste this into another LaTeX file where you 
% want your title page.
% OR
% 2) Remove everything outside the \begin{titlepage} and \end{titlepage} and 
% move this file to the same directory as the LaTeX file you wish to add it to. 
% Then add \input{./title_page_1.tex} to your LaTeX file where you want your
% title page.
%
%%%%%%%%%%%%%%%%%%%%%%%%%%%%%%%%%%%%%%%%%
%\title{Title page with logo}
%----------------------------------------------------------------------------------------
%	PACKAGES AND OTHER DOCUMENT CONFIGURATIONS
%----------------------------------------------------------------------------------------

\documentclass[12pt]{article}

\usepackage[francais]{babel}
\usepackage[utf8x]{inputenc}
\usepackage[T1]{fontenc}

\usepackage{amsmath}
\usepackage{graphicx}
\usepackage{enumerate}

% Define new command
\newcommand{\HRule}{\rule{\linewidth}{0.5mm}}

\newcommand{\crt}{\emph{Nexys 4 DDR\ }}


\begin{document}

\begin{titlepage}

\center % Center everything on the page
 
%----------------------------------------------------------------------------------------
%	HEADING SECTIONS
%----------------------------------------------------------------------------------------

\textsc{\LARGE Université Pierre et Marie Curie}\\[1.5cm] % Name of your university/college
\textsc{\Large Major Heading}\\[0.5cm] % Major heading such as course name

%----------------------------------------------------------------------------------------
%	TITLE SECTION
%----------------------------------------------------------------------------------------

\HRule \\[0.4cm]
{ \huge \bfseries Rapport de pré-soutenance}\\[0.4cm] % Title of your document
{ \huge \bfseries Développement sur FPGA \\d'un système d'aiguillage
  \\pour centale DCC sur train miniature}\\[0.4cm] % Title of your document
\HRule \\[1.5cm]
 
%----------------------------------------------------------------------------------------
%	AUTHOR SECTION
%----------------------------------------------------------------------------------------

\begin{minipage}{0.4\textwidth}
\begin{flushleft} \large
\emph{\'Etudiant:}\\
Maxime \textsc{AYRAULT} 3203694 % Your name
\end{flushleft}
\end{minipage}
~
\begin{minipage}{0.4\textwidth}
\begin{flushright} \large
\emph{Encadrant:} \\
Julien \textsc{DENOULET} % Supervisor's Name
\end{flushright}
\end{minipage}\\[2cm]

%----------------------------------------------------------------------------------------
%	DATE SECTION
%----------------------------------------------------------------------------------------

{\large \today}\\[2cm] % Date, change the \today to a set date if you want to be precise

%----------------------------------------------------------------------------------------
%	LOGO SECTION
%----------------------------------------------------------------------------------------

%%\begin{figure}
%%  \subfigure[]{\includegraphics[scale=0.2\textwidth]{logo.png}} 
%%\end{figure} 
\includegraphics[width=0.2\textwidth]{logo.png}

%----------------------------------------------------------------------------------------

\vfill % Fill the rest of the page with whitespace

\end{titlepage}



\begin{abstract}
Your abstract here.
\end{abstract}

\section{Introduction}
\label{sec:introduction}

\subsection{Contexte et encadrement}

Depuis plusieurs années, le laboratoire xxx développe un projet de
gestion de maquettes de trains. Ce projet se base sur une
\emph{centrale DCC}. Cette centrale permet de recevoir la position des
trains et des aiguilles et d'envoyer des commandes en utilisant le
\emph{protocole DCC}.

Ce projet est réalisé seul et est encadré par le Responsable de la
valuer FPGA. Ce projet se déroule sur une durée de 6 semaines.
Bla bla bla...

\subsection{Objectifs}

L'objectif de mon projet consiste, dans un premier temps, à porter la
\emph{centrale DCC} implementée par d'autres étudiants sur une nouvelle
carte matériel \emph{FPGA}. La version courante de la centrale tourne sur
une carte \emph{Spartan 6}, elle est remplacée par notre \crt pour Bla
bla bla...
Dans un second temps, mon projet consiste à ajouter la commande de
des aiguillages.

Je propose par commencer par une gestion manuelle des aiguillages avec
les différents switchs de la carte sans vérification de sécurité. Puis
une fois cela fait, je vais implémenter par une gestion des
enclenchements ferroviaire grâce aux differents capteurs présents sur
les rails. Ceci permet de garantir qu'un train ne devra pas pouvoir
changer de voie que si aucun autre train ne se trouve sur le chemin
qu'il veut parcourir. 

Les domaines de compétences requis sont multiples; la connaissance
du langage VDHL et de la plateforme \emph{FPGA} pour l'implémentation
de la centrale DCC et la connaissance de la gestion des enclenchements
ferroviaires.


\newpage
\section{Aspects techniques}
\label{sec:asp_tech}

\subsection{Le protocole DCC}
\label{sec:dcc}
Ce protocole est un protocole standardisé  communiquer entre notre carte
\emph{FPGA} et les différents trains. Il utilise une suite de commandes envoyées sur les rails  
jusqu'aux différents trains qui agisent en fonction de ce qu'ils recoivent.

Bla bla bla...

\subsection{La logique d'enclenchement}
\label{sec:log_ixl}

Un système ferroviaire est composé de plusieurs systèmes :
\begin{itemize}
  \item Les équipements de voie (rails, aiguillages...) et le matériel
    roulant. C'est la partie visible des passagers
  \item Le Poste de Commande ferroviaire qui permet à un opérateur de
    visualiser, en temps réel, l'état du système (position des trains,
    position des aiguilles...
  \item La logique d'enclenchement qui assure la sécurité du système
    ferroviaire. Il est placé entre le Poste de Commande ferroviaire
    et les équipements de voie. Il interdit les commandes lorsque les
    conditions incompatibles avec la sécurité. 
\end{itemize}

La figure suivante présente les relations entre les différents
systèmes.


\subsection{Architecture générale}
\label{sec:archi}
Bla bla bla...



\subsection{Outils}
\label{sec:outils}

Lors de ce projet, je vais devoir utiliser différents outils :
\begin{itemize}
  \item La carte \crt comme plateforme de développement
  \item La locomotive \emph{Jouef ``Fret SNCF``}\cite{Jouef}  et les capteurs de
    position et des aiguillages
  \item Le protocole DCC \cite{DCC}
  \item Le logiciel Vivado comme \emph{IDE} ?????
\end{itemize}

ainsi que plusieurs langages:
\begin{itemize}
  \item \emph{VHDL}\cite{VHDL} comme language de description pour mes différentes
    \emph{IP}
  \item \emph{GIT}\cite{GIT} pour la gestion de configuration des logiciels
  \item \emph{\LaTeX}\cite{LATEX} pour la rédaction de la documentation
  \item Le langage \emph{Ocaml}\cite{OCAML} pour la génération de la logique d'enclenchement
\end{itemize}


\subsection{Validation}
\label{sec:valid}

Afin de tester et valider les differentes étapes de mon projet je vais
devoir faire des \emph{bancs de tests}, des simulations ainsi que des
expérimentations qui pourront me permettre de valider mes essais.

Il y a d'ailleurs plusieurs scénarios que je vais devoir réaliser qui
me permettrons de tester de façon réelle le fonctionnement de mes travaux.

En voici quelques un:

\begin{enumerate}[A]
  \item Sans Capteurs
  \begin{itemize}
    \item 1 train doit passer de la voie ``A`` à la voie ``B``
    \item 2 trains sur la voie ``A``, un seul doit passer sur la voie
       ``B``
  \end{itemize}

  \item Avec Capteurs
  \begin{itemize}
    \item 1 train doit passer de la voie ``A`` à la voie ``B``
    \item 2 trains sur la voie ``A``, un seul doit passer sur la voie ``B``
    \item 2 trains sur la voie 1 avec 1 seul qui doit passer de voie 1 à voie 2
    \item 1 train A sur voie 2 qui s'arrête dans la zone aiguillage.
    \item 1 train B sur voie 1 qui veut passer en voie 2 --> Pas
       possible. Redémarrage train A, vérification du changement de
       voie du train B.
  \end{itemize}
\end{enumerate}


\newpage
\section{Organisation du projet}
\label{sec:org_proj}

\subsection{Activités du projet}
\label{sec:activ}


\subsection{Planning prévisionnel}
\label{sec:planning}


\subsection{Avancement}
\label{sec:avanc}


\newpage
\bibliographystyle{plain}
\bibliography{biblio}

\end{document}
