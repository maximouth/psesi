%%%%%%%%%%%%%%%%%%%%%%%%%%%%%%%%%%%%%%%%%
% University Assignment Title Page 
% LaTeX Template
% Version 1.0 (27/12/12)
%
% This template has been downloaded from:
% http://www.LaTeXTemplates.com
%
% Original author:
% WikiBooks (http://en.wikibooks.org/wiki/LaTeX/Title_Creation)
%
% License:
% CC BY-NC-SA 3.0 (http://creativecommons.org/licenses/by-nc-sa/3.0/)
% 
% Instructions for using this template:
% This title page is capable of being compiled as is. This is not useful for 
% including it in another document. To do this, you have two options: 
%
% 1) Copy/paste everything between \begin{document} and \end{document} 
% starting at \begin{titlepage} and paste this into another LaTeX file where you 
% want your title page.
% OR
% 2) Remove everything outside the \begin{titlepage} and \end{titlepage} and 
% move this file to the same directory as the LaTeX file you wish to add it to. 
% Then add \input{./title_page_1.tex} to your LaTeX file where you want your
% title page.
%
%%%%%%%%%%%%%%%%%%%%%%%%%%%%%%%%%%%%%%%%%
%\title{Title page with logo}
%----------------------------------------------------------------------------------------
%	PACKAGES AND OTHER DOCUMENT CONFIGURATIONS
%----------------------------------------------------------------------------------------

\documentclass[12pt]{article}

\usepackage[francais]{babel}
\usepackage[utf8x]{inputenc}
\usepackage[T1]{fontenc}
\usepackage{color}

\usepackage{amsmath}
\usepackage{graphicx}
\usepackage{enumerate}
\usepackage{url}

% Define new command
\newcommand{\HRule}{\rule{\linewidth}{0.5mm}}

\newcommand{\crt}{\emph{Nexys 4 DDR\ }}
%\def\thesubsection{\alph{section}}

\begin{document}

\begin{titlepage}

\center % Center everything on the page
 
%----------------------------------------------------------------------------------------
%	HEADING SECTIONS
%----------------------------------------------------------------------------------------

\textsc{\LARGE Universit\'e Pierre et Marie Curie}\\[1.5cm] % Name of your university/college
\textsc{\Large PSESI}\\[0.5cm] % Major heading such as course name

%----------------------------------------------------------------------------------------
%	TITLE SECTION
%----------------------------------------------------------------------------------------

\HRule \\[0.4cm]
{ \huge \bfseries Projet Centrale DCC}\\[0.4cm] % Title of your document
{ \huge \bfseries Document Définition Interface et Logique Enclenchement}\\[0.4cm] % Title of your document
\HRule \\[1.5cm]
 
%----------------------------------------------------------------------------------------
%	AUTHOR SECTION
%----------------------------------------------------------------------------------------

\begin{minipage}{0.4\textwidth}
\begin{flushleft} \large
\emph{\'Etudiant:}\\
Maxime \textsc{AYRAULT} 3203694 % Your name
\end{flushleft}
\end{minipage}
~
\begin{minipage}{0.4\textwidth}
\begin{flushright} \large
\emph{Encadrant:} \\
Julien \textsc{DENOULET} % Supervisor's Name
\end{flushright}
\end{minipage}\\[2cm]

%----------------------------------------------------------------------------------------
%	DATE SECTION
%----------------------------------------------------------------------------------------

{\large \today}\\[2cm] % Date, change the \today to a set date if you want to be precise

%----------------------------------------------------------------------------------------
%	LOGO SECTION
%----------------------------------------------------------------------------------------

%%\begin{figure}
%%  \subfigure[]{\includegraphics[scale=0.2\textwidth]{logo.png}} 
%%\end{figure} 
\includegraphics[width=0.2\textwidth]{logo.png}

%----------------------------------------------------------------------------------------

\vfill % Fill the rest of the page with whitespace

\end{titlepage}



%\begin{abstract}
%Your abstract here.
%\end{abstract}

\section{Introduction}
\label{sec:introduction}
Présentation des 3 interfaces~:
\begin{itemize}
\item Interface Homme Machine
\item Interface Centrale DCC et le circcuit ferroviaire
\item Interface interne Centrale DCC - Moteur IXL
\end{itemize}

et présentation des règles de la logique d'enclenchement.


\section{Interface Homme Machine}
\label{sec:int-dcc}

\subsection{\underline{Interface commande train}}
\label{sec:ihm-train}

bla bla bla...

\subsection{\underline{Interface itinéraire}}
\label{sec:ihm_iti}

bla bla bla...



\section{Interface Centrale DCC - Train, aiguillage et capteurs}
\label{sec:int_dcc}

\subsection{\underline{Interface Train}}
\label{sec:int_train}

bla bla bla...-+

\subsection{\underline{Interface Aiguillage}}
\label{sec:int_aig}

bla bla bla...

\subsection{\underline{Interface Capteurs}}
\label{sec:int_cap}

bla bla bla...


\newpage
\section{\underline{Interface Centrale DCC - Moteur IXL}}
\label{sec:int_ixl}   
Bla bla bla...


\newpage
\section{\underline{Logique Enclenchement}}
\label{sec:log_enc}

\subsection{\underline{Circuit de Voie}}
\label{sec:CdV}

Un Circuit de Voie est la section de voie se trouvant entre 2 capteurs
consécutifs. Les capteurs de présence ne permettent pas de savoir dans
quel sens va le train. Ils permettent uniquement de lire un code barre
placé sous le train.  

\subsection{\underline{Zone d'aiguille}}
\label{sec:aig}

Bla bla bla...


\subsection{\underline{Itinéraire}}
\label{sec:iti}

Bla bla bla...

\subsection{\underline{zone d'aiguille}}
\label{sec:aig}

Bla bla bla...



\newpage

\bibliographystyle{plain}
\bibliography{biblio}

\end{document}
