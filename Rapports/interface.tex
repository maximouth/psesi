%%%%%%%%%%%%%%%%%%%%%%%%%%%%%%%%%%%%%%%%%
% University Assignment Title Page 
% LaTeX Template
% Version 1.0 (27/12/12)
%
% This template has been downloaded from:
% http://www.LaTeXTemplates.com
%
% Original author:
% WikiBooks (http://en.wikibooks.org/wiki/LaTeX/Title_Creation)
%
% License:
% CC BY-NC-SA 3.0 (http://creativecommons.org/licenses/by-nc-sa/3.0/)
% 
% Instructions for using this template:
% This title page is capable of being compiled as is. This is not useful for 
% including it in another document. To do this, you have two options: 
%
% 1) Copy/paste everything between \begin{document} and \end{document} 
% starting at \begin{titlepage} and paste this into another LaTeX file where you 
% want your title page.
% OR
% 2) Remove everything outside the \begin{titlepage} and \end{titlepage} and 
% move this file to the same directory as the LaTeX file you wish to add it to. 
% Then add \input{./title_page_1.tex} to your LaTeX file where you want your
% title page.
%
%%%%%%%%%%%%%%%%%%%%%%%%%%%%%%%%%%%%%%%%%
%\title{Title page with logo}
%----------------------------------------------------------------------------------------
%	PACKAGES AND OTHER DOCUMENT CONFIGURATIONS
%----------------------------------------------------------------------------------------

\documentclass[12pt]{article}

\usepackage[francais]{babel}
\usepackage[utf8x]{inputenc}
\usepackage[T1]{fontenc}
\usepackage{color}

\usepackage{amsmath}
\usepackage{graphicx}
\usepackage{enumerate}
\usepackage{url}
\usepackage{listings}

\usepackage{xcolor}
\colorlet{keyword}{blue!100!black!80}
\colorlet{comment}{green!90!black!90}
\lstdefinestyle{vhdl}{
   language     = VHDL,
   basicstyle   = \ttfamily,
   keywordstyle = \color{keyword}\bfseries,
   commentstyle = \color{comment}
}

\lstdefinestyle{equ}{
   basicstyle   = \ttfamily,
   keywordstyle = \color{keyword}\bfseries,
   commentstyle = \color{comment}
}

% Define new command
\newcommand{\HRule}{\rule{\linewidth}{0.5mm}}

\newcommand{\crt}{\emph{Nexys 4 DDR\ }}
%\def\thesubsection{\alph{section}}

\begin{document}

\begin{titlepage}

\center % Center everything on the page
 
%----------------------------------------------------------------------------------------
%	HEADING SECTIONS
%----------------------------------------------------------------------------------------

\textsc{\LARGE Universit\'e Pierre et Marie Curie}\\[1.5cm] % Name of your university/college
\textsc{\Large PSESI}\\[0.5cm] % Major heading such as course name

%----------------------------------------------------------------------------------------
%	TITLE SECTION
%----------------------------------------------------------------------------------------

\HRule \\[0.4cm]
{ \huge \bfseries Projet Centrale DCC}\\[0.4cm] % Title of your document
{ \huge \bfseries Document Définition Interface et Logique Enclenchement}\\[0.4cm] % Title of your document
\HRule \\[1.5cm]
 
%----------------------------------------------------------------------------------------
%	AUTHOR SECTION
%----------------------------------------------------------------------------------------

\begin{minipage}{0.4\textwidth}
\begin{flushleft} \large
\emph{\'Etudiant:}\\
Maxime \textsc{AYRAULT} 3203694 % Your name
\end{flushleft}
\end{minipage}
~
\begin{minipage}{0.4\textwidth}
\begin{flushright} \large
\emph{Encadrant:} \\
Julien \textsc{DENOULET} % Supervisor's Name
\end{flushright}
\end{minipage}\\[2cm]

%----------------------------------------------------------------------------------------
%	DATE SECTION
%----------------------------------------------------------------------------------------

{\large \today}\\[2cm] % Date, change the \today to a set date if you want to be precise

%----------------------------------------------------------------------------------------
%	LOGO SECTION
%----------------------------------------------------------------------------------------

%%\begin{figure}
%%  \subfigure[]{\includegraphics[scale=0.2\textwidth]{logo.png}} 
%%\end{figure} 
\includegraphics[width=0.2\textwidth]{logo.png}

%----------------------------------------------------------------------------------------

\vfill % Fill the rest of the page with whitespace

\end{titlepage}



%\begin{abstract}
%Your abstract here.
%\end{abstract}

\section{Introduction}
\label{sec:introduction}
Présentation des 3 interfaces~:
\begin{itemize}
\item Interface Homme Machine
\item Interface Centrale DCC et le circcuit ferroviaire
\item Interface interne Centrale DCC - Moteur IXL
\end{itemize}

et présentation des règles de la logique d'enclenchement.


\section{Interface Homme Machine}
\label{sec:int-dcc}
Sur la carte
\begin{itemize}
\item Bouton gauche -> changement du paramètre
\item Bouton droit  -> changement de la valeur du paramètre
\end{itemize}

\hrulefill

Dans le design
\begin{enumerate}
\item en sortie :
  \begin{itemize}  
  \item les 7 selecteurs de segment Ca -> Cg
  \item les 8 selecteurs de digit   An (7 downto 0)
  \item octet adresse \textbf{\emph{(addr or aigui)}}  dans ADD
  \item octet speed \textbf{\emph{(speed)}}  dans SPEED
  \item octet fonction \textbf{\emph{(feat)}}  dans FCTN    
  \end{itemize}  
\item en entrée :
  \begin{itemize}  
  \item une horloge à 100 MHz
  \item un reset synchrone actif haut 
  \item la valeur du bouton gauche BTNL
  \item la valeur du bouton droit BTNR    
  \end{itemize}  
\end{enumerate}

\subsection{\underline{Interface commande train}}
\label{sec:ihm-train}

\bigskip

8 vitesses
\begin {enumerate}
\item vitesse 0 -> \emph{``01100000''}
\item vitesse 1 -> \emph{``01100010''}
\item vitesse 2 -> \emph{``01100100''}
\item vitesse 3 -> \emph{``01101000''}
\item vitesse 4 -> \emph{``01101010''}
\item vitesse 5 -> \emph{``01110000''}
\item vitesse 6 -> \emph{``01110100''}
\item vitesse 7 -> \emph{``01111111''}
\end {enumerate}  

\bigskip

6 adresses de train
\begin {enumerate}
\item adresse 0 -> \emph{``00000000''}  
\item adresse 1 -> \emph{``00000001''}
\item adresse 2 -> \emph{``00000010''}
\item adresse 3 -> \emph{``00000011''}
\item adresse 4 -> \emph{``00000100''}
\item adresse 5 -> \emph{``00000101''}
\item adresse 6 -> \emph{``00000110''}
\end {enumerate}  

\bigskip

8 adresses de d'aiguillage
\begin {enumerate}
\item elles ne sont pas encore définies. mais de 10 à 18 pour faire simple.
\end {enumerate}  

\bigskip

X fonctions
\begin {enumerate}
\item phares on     -> \emph{``10010000''}
\item bruit moteur  -> \emph{``10000001''}
\item klaxon 1 long -> \emph{``10000100''}
\item klaxon 1 bref -> \emph{``10100100''}
\item bruit du vent -> \emph{``10100001''}
\end {enumerate}  

\bigskip

\subsection{\underline{Interface itinéraire}}
\label{sec:ihm_iti}

bla bla bla...

\newpage

\section{Interface Centrale DCC - Train, aiguillage et capteurs}
\label{sec:int_dcc}

\subsection{\underline{Interface Train}}
\label{sec:int_train}

\bigskip
Trames envoyéees 4 par 4 séparées chacunes par 400 us avec valeur 0\linebreak
1er  trame IDLE\linebreak
2eme trame Vitesse\linebreak
3eme trame Action\linebreak
4eme trame IDLE\linebreak
\medskip

Une fois une trame envoyé, la repeter toutes les 55 ms\linebreak
si aucune trame de chargée envoyer des trames idle\linebreak
adresse => ``11111111''\linebreak
donée   => ``00000000''\linebreak
CRC     => ``11111111''\linebreak
\medskip

Parametre à envoyer stocké dans \emph{Master}\hfill\linebreak
-> à chaques pression sur le bouton central changement de la valeur\linebreak
des paramètres à envoyer, Les paramètre viennent de l'IHM et\linebreak
remplacent la valeur IDLE et sont envoyés en continue tant que pas de\linebreak
nouvelle prssion sur bouton C.\linebreak
\bigskip

\subsection{\underline{Interface Aiguillage}}
\label{sec:int_aig}

bla bla bla

\subsection{\underline{Interface Capteurs}}
\label{sec:int_cap}

bla bla bla...


\newpage
\section{\underline{Interface Centrale DCC - Moteur IXL}}
\label{sec:int_ixl}   

La logique d'enclenchement est basée sur les notions de signaux, de circuit de voie et d'itinéraire ferroviaire.

Un \emph{signal} est le moyen de communication entre le conducteur du train et la logique d'enclenchement. 
Il permet d'indiquer au conducteur l'état de la voie se trouvant devant le train (par définition un 
train ne recule jamais, il est nécessaire de changer de cabine de conduite pour changer
la direction du train). Il existe beaucoup de signaux ferroviaire. Les trois plus importants sont~:
\begin{itemize}
\item signal \emph{rouge} qui signifie que le train doit s'arrêter devant le signal car la
voie devant lui n'est pas sûre (présence d'un autre train, aiguillage en train d'être commandé, panne 
matériel...)
\item signal \emph{jaune} qui signifie qu'il y a un danger à proximité. Le train peut passer le signal
mais à vitesse réduite seulement et doit être capable de s'arrêter sur une courte distance. Ceci permet 
de fluidifier le traffic.
\item signal \emph{vert} qui signifie que la voie devant le train est libre de tout danger et que le train 
peut poursuivre sa route en respectant les vitesses limites.
\end{itemize}

L'objectif des opérateurs consiste à toujours présenter un signal vert devant un train de façon 
à ce que le train n'est jamais à s'arrêter à cause d'un danger.


\medskip
Le projet implémente le signal rouge et le signal vert. Le signal jaune ne peut pas être implémenté sans conducteur 
dans le train. Il n'y a pas de signaux sur le circuit ferroviaire, mais la notion de signaux sera quand même
implémentée. Un signal \emph{virtuel} sera créé pour chaque circuit de voie. Il autorisera ou pas l'entrée d'un train dans 
le circuit de voie auquel il est associé.
Ceci permettra de commander les trains. 

\medskip
Un \emph{Ciruit de Voie} (CdV) représente la plus petite partie de voie sur lequel il
est possible de détecter la présence d'un train. Pour un train réel, la détection de la 
présence d'un train s'effectue par les essieux du train. En effet, les essieux du train sont conducteur
(métal), ils peuvent servir comme un \emph{interupteur} entre les 2 rails de la voie. Si une partie du 
train est présent alors les 2 rails sont au même potentiel (le circuit est dit fermé), sinon les rails sont 
a des potentiels différents (le circuit est dit ouvert). 

\medskip
Pour le projet, un Circuit de voie sera délimité par 2 capteurs de passage(sauf en zone d'aiguille
où le circuit de voie sera délimité par les capteurs de chaque branche). L'état d'occupation du 
CdV sera fourni pas le sens de parcours identifié par le capteur pour chaque train.

\medskip
Un \emph{itinéraire ferroviaire} représente une portion de voie entre 2 signaux. Il contient les mouvements
d’aiguilles nécessaires pour que le train parcourt le chemin entre le signal de
départ et le signal de sortie de l'itinéraire. Dans le système d'enclenchement, un itinéraire
est représenté par une machine à états. Les états possibles sont:
\begin{itemize}
\item \emph{commandé}. L’opérateur demande au train de parcourir l’itinéraire
choisi. Il a par conséquent demandé la formation de l’itinéraire au
système d’enclenchement. Si les conditions sur les aiguilles à manoeuvrer
pour former l’itinéraire sont sûres, le système d’enclenchement
passe à l’état de formation de l’itinéraire sinon il est rejeté.
\item \emph{formé}. Les aiguilles sont positionnées dans la bonne direction. Il n’est
plus possible de bouger les aiguilles tant que le train se situe dans une
certaine zone.
\item \emph{détruit}. Le train a parcouru l’itinéraire (il a passé le signal de sortie), l'itinéraire peut
 être détruit. Ou l'opérateur a demandé la destruction de l'itinéraire (en cas d'erreur par exemple)
Dans ce cas, si le train ne se trouve pas trop proche du signal d'entrée alors l'itinéraire peut 
être détruit sinon la demande de destruction est rejetée. 
Ceci permet d'autoriser le parcours sur les aiguilles et la zone d’un autre train. 
\end{itemize}


L'interface entre la Centrale DCC et le Moteur IXL est bi-directionnelle. 
Dans le sens Centrale vers IXL, les informations envoyées sont~:
\begin{itemize}
\item les commandes d'itinéraire
\item l'information des capteurs de passage
\item l'état des aiguilles (gauche/droite)
\end{itemize}

Dans le sens IXL vers Centrale, les informations envoyées sont~:
\begin{itemize}
\item les commandes des aiguilles
\item l'état des itinéraires (commandé/formé/détruit)
\item l'état des signaux (rouge/vert)
\end{itemize}

\subsection{\underline{Commande itinéraire}}
\label{sec:ixl_iti}

bla bla bla...

\subsection{\underline{\'Information capteur de passage}}
\label{sec:ixl_cdv}



\medskip
La déclaration VHDL
\begin{lstlisting}[style=vhdl]
  type train_addr is 7 downto 0 ;
  type train_dir is bit ;

  type sensor is
   record
       addr : train_addr;
       dir  : train_dir;
   end record;

  type SE_state is array (31 downto 0) of sensor;
\end{lstlisting}


\subsection{\underline{\'Etat des aiguillages}}
\label{sec:ixl_aig}

L'état d'une agiuille est déterminé par la combinaison entre les 2 capteurs d'aiguilles.
Une aiguille peut avoir 3 états:
\begin{itemize}
\item \emph{gauche} lorsque le contact gauche est à l'état 1
\item \emph{droite} lorsque le contact droit est à l'état 1
\item \emph{en mouvement} lorsqu'aucun des 2 contacts ne sont à 1
\end{itemize}

Il n'y a pas de capteur fournissant l'état d'une aiguille sur la maquette. 
Par conséquent, l'état d'une aiguille correspondra à l'état de la dernière commande envoyée
 à l'aiguille. L'état \emph{en mouvement} ne sera pas considéré.
 
L'état d'une aiguille sera représenté par 1 bit avec la convention 
\begin{itemize}
\item 0 = droite
\item 1 = gauche
\end{itemize}
 
Il y a 4 aiguilles sur le circuit. L'indice du tableau correspond au numéro de l'aiguille.

\medskip
La déclaration VHDL
\begin{lstlisting}[style=vhdl]
  type Sw_state is array (7 downto 0) of bit;  
\end{lstlisting}


\subsection{\underline{\'Etat des itinéraires}}
\label{sec:st_iti}

Les états possibles pour un itinéraire sont; commandé, formé, détruit et rejeté. Il sera représenté
par 2 bits avec la conversion
\begin{itemize} 
\item 00 = commandé
\item 01 = formé
\item 10 = détruit
\item 11 = rejeté
\end{itemize}

Il y a 32 itinéraires possibles. L'indice dans le tableau correspond au numéro d'itinéraire.

\medskip
La déclaration VHDL
\begin{lstlisting}[style=vhdl]
  type Ro_state_t is 3 downto 0 ;

  type Ro_state is array (31 downto 0) of Ro_state_t;  
\end{lstlisting}


\subsection{\underline{Commande des aiguillages}}
\label{sec:cmd_aig}

bla bla bla...

\subsection{\underline{\'Etat des signaux}}
\label{sec:st_sig}

bla bla bla...

  
\subsection{\underline{\'Etat des Circuits de Voie}}
\label{sec:st_sig}

Chaque circuit de voie possède 2 états; \emph{Libre} ou
\emph{Occupé}. De façon à pouvoir effectuer la gestion des trains, il
est nécessaire d'identifier quel train occupe le circuit de voie.
Il y a 6 trains au maximum. Les états sont codés come suit~:
\begin{itemize}
  \item libre : 0x07
  \item occupé : numéro du train 0x01 à 0x06
\end{itemize}  

Il y a 20 circuits de voie sur la maquette de train. L'indice du tableau 
correspond au numéro du Circuit de Voie.




\newpage
\section{\underline{Logique Enclenchement}}
\label{sec:log_enc}

Cette secion présente la logique d'enclenchement, sa justification et les 
conversions de nommage utilisées dans les équations.

\subsection{\underline{Circuit de Voie}}
\label{sec:CdV}

Un Circuit de Voie est la section de voie se trouvant entre 2 capteurs de présence
consécutifs. Un capteur de présence permettent de connaître, le numéro du train et
son sens de déplacement. Le sens de déplacement va permettre de savoir quel 
Circuit de voie occupé par le train. Deux directions sont arbitrairement déclarées
pour chaque capteur; Up et Down.

Les convention de nommage des éléments~:
\begin{itemize}
\item SE\_Up\_nn, 1 lorsqu'un train vient de passer le capteur numéro nn dans le sens Up
\item SE\_Do\_nn, 1 lorsqu'un train vient de passer le capteur numéro nn dans le sens Down
\item TC\_nn, 1 lorsque le Circuit de Voie est libre (aucun train présent sur le CdV), 0 lorsque 
le circuit de voie est occupée. Cette logique va dans le sens de la sécurité, en cas
de perte de courant ou de dysfonctionnement, il n'est pas possible de savoir si
le Circuit de VOie est occupé ou libre, donc on choisit la valeur la plus sécuritaire;
Occupé même si aucun train n'est présent.
\end{itemize}

\'A l'initialisation, tous les Circuits de Voie sont libres. 

\medskip
Le Circuit de voie TC\_nn est occupé si une des condition suivantes est vérifiée~:
\begin{itemize}
\item le capteur SE\_Up\_nn passe à 1, un train vient de rentrer dana le sens Up
\item le capteur SE\_Do\_mm passe à 1, un train vient de rentrer dans le sens Down
\item le Circuit de Voie est occupé et le train n'est pas sorti dans le sens Up ou Down
\end{itemize}

Il faut inverser cette logique car lorsque le Circuit de voie est occupé, on cherche à
avoir une valeur 0.

\medskip
L'équation générale d'occupation d'un Circuit de Voie est~:
\begin{lstlisting}
  TC_nn <- NOT SE_Up_nn AND NOT SE_Do_mm 
           AND (TC_nn OR SE_Do_nn OR SE_Up_mm);
\end{lstlisting}


\subsection{\underline{Signaux}}
\label{sec:esp}

Il existe deux types de signaux; les signaux d'espacement qui garantissent
que 2 trains ne vont pas se percuter et les sinaux de manoeuvre qui garantissent
qu'un train ne va pas parcourir une aiguille qui n'est pas contrôlée.

Les convention de nommage des éléments~:
\begin{itemize}
\item SI\_nn, 1 lorsque le signal est Vert, 0 pour afficher un signal Rouge
\end{itemize}

\'A l'initialisation, tous les signaux sont au Vert. 

Un Signal d'espacement passe~:
\begin{itemize}
\item au Rouge, lorsque le signal d'espacement ou le signal de manoeuvre est au Rouge
\item au Vert, lorsque le signal d'espacement et le signal de manoeuvre sont au Vert
\end{itemize}


\medskip
L'équation générale d'un signal est~:
\begin{lstlisting}
  SI_nn <- SI_SP_nn AND SI_IX_nn;
\end{lstlisting}


\subsection{\underline{Signal d'espacement des trains}}
\label{sec:esp}

Il est dangereux d'avoir 2 trains dans le même Circuit de Voie car rien  ne 
permet de garantir que les 2 trains ne vont pas se percuter. Il n'est pas certain
qu'un conducteur puisse arrêter le train devant un signal car la distance de
freinage entre le signal et le Circuit de Voie à protéger n'est pas toujours 
suffisante. C'est pourquoi, il est nécessaire de toujours avoir un Circuit de Voie 
libre entre 2 trains. Ce Citcuit de Voie est nommé, \emph{CdV tampon}. 
Dans le cas de 2 trains qui se suivent, il est nécessaire que la longueur du Circuit 
de Voie soit plus grande que la distance d'arrêt du train suiveur.
Dans le cas de 2 trains en sens inverse, il est nécessaire de s'assurer que la
longueur du Circuit de Voie tampon est égale à 2 fois la longueur d'arrêt 
d'un train. 

L'entrée dans un Circuit de Voie est protégée par un signal dit \emph{d'espacement}.

Les convention de nommage des éléments~:
\begin{itemize}
\item SI\_SP\_nn, 1 lorsque le signal est Vert, 0 pour afficher un signal Rouge
\end{itemize}

\'A l'initialisation, tous les signaux sont au Vert. 

Un Signal d'espacement passe~:
\begin{itemize}
\item au Rouge, lorsque le TC\_nn est occupé ou le TC\_mm est occupé
\item au Vert, lorsque les TC\_nn et TC\_mm sont libres.
\end{itemize}

\medskip
L'équation générale d'un signal d'espacement est~:
\begin{lstlisting}
  SI_SP_nn <- TC_nn AND TC_mm;
\end{lstlisting}


\subsection{\underline{Signal de manoeuvre}}
\label{sec:esp}

Un signal de manoeuvre garantit qu'un train peut traverser une zone d'aiguillage 
sans danger. C'est à dire que les aiguilles de la zones sont toutes enclenchées.




\subsection{\underline{Zone d'aiguille}}
\label{sec:aig}

Une aiguille peut être commandée si elle n'est pas enclenchée et si il n'y a pas de train
sur la zone 


\subsection{\underline{Itinéraire}}
\label{sec:iti}

Bla bla bla...

\subsection{\underline{zone d'aiguille}}
\label{sec:aig}

Bla bla bla...



\newpage

\bibliographystyle{plain}
\bibliography{biblio}

\end{document}
